\documentclass[a4paper,10pt]{article}
\usepackage[]{graphicx}
\usepackage[]{times}
\usepackage{geometry}
\usepackage{framed}

\geometry{verbose,a4paper,tmargin=1.5cm,bmargin=2cm,lmargin=2cm,rmargin=2cm}
\renewcommand{\baselinestretch}{1.2}

\title{Gici RKLT Manual \\ \small (version 1.1)}

\author{
GICI group \vspace{0.1cm} \\
\small Department of Information and Communications Engineering \\
\small Universitat Aut{\`o}noma Barcelona \\
\small http://www.gici.uab.es  -  http://gici.uab.cat/GiciWebPage/downloads.php \\
}

\date{January 2010}

\begin{document}
\maketitle

\section{Description}

This software is an implementation of the Reversible Karhunen-Lo\^eve Transform (RKLT) to be used as spectral decorrelator.
In addition it features many state-of-the-art techniques for spectral decorrelation:

\begin{itemize}
\item Lossy Karhunen-Lo\^eve Transform.
\item Matrix factorizations for reversible integer mapping~\cite{HS01,GS04}.
\item Covariance Subsampling~\cite{PTM+06,PTM+07a}.
\item Clustered and Multi-level transforms~\cite{BS09,BS10}.
\end{itemize}

\section{Requirements}

This software is programmed in Java, so you might need a JAVA Runtime Environment(JRE) to run this application.
We have used SUN JAVA 1.5. 

\begin{description}
\item[JAI] The Java Advanced Imaging (JAI) library is used to load and save images in formats
other than raw or pgm. The JAI library can be freely downloaded from \emph{http://java.sun.com}.
\textbf{Note:} You don't need to have this library installed in order to compile the source code.

\item[GSL] Eigendecomposition functions are from the GNU Scientific Library (GSL) and have been translated into Java.
The authors of the of original code are Gerard Jungman and Brian Gough. (see source files for details)
\end{description}

\section{Usage}

The application is provided in a single file, a jar file (\emph{dist/rklt.jar}), that contains the application.
Along with the application, the source code is also provided. If you need to rebuild the jar file, you can use the \texttt{ant} command.

To launch the application you can use the following command: 

\begin{framed}
\texttt{\$ java -Xmx1200m -jar dist/rklt.jar --help}
\end{framed}

In a GNU/Linux environment you can also use the shell script \texttt{rklt} situated at the root of the RKlt directory. 

\begin{framed}
\texttt{\$ ./rklt --help}
\end{framed}

Some examples of usage are provided below:

\begin{itemize}
\item Coding and decoding an image with the Reversible KLT:
\begin{framed}%
\vspace{-1em}%
\begin{verbatim}
$ ./rklt -i inptfile-16bpppb-bigendian-224x512x512.raw \
         -o rkltfile-16bpppb-bigendian-224x512x512.raw \
         -ig 224 512 512 3 0 -og 224 512 512 4 0 \
         -ti side-information.file -D 0 -d 0

$ ./rklt -i rkltfile-16bpppb-bigendian-224x512x512.raw \ 
         -o outpfile-16bpppb-bigendian-224x512x512.raw \
         -ig 224 512 512 4 0 -og 224 512 512 3 0 \
         -ti side-information.file -D 0 -d 1 
\end{verbatim}%
\vspace{-1em}%
\end{framed}

\item Forward transform with covariance subsampling enabled:
\begin{framed}%
\vspace{-1em}%
\begin{verbatim}
$ ./rklt -i inptfile-16bpppb-bigendian-224x512x512.raw \
         -o rkltfile-16bpppb-bigendian-224x512x512.raw \
         -ig 224 512 512 3 0 -og 224 512 512 4 0 \
         -ti side-information.file -D 0 -d 0 \
         -es 0.01
\end{verbatim}%
\vspace{-1em}%
\end{framed}

\item Using the dynamic structure defined in~\cite{BS10}:
\begin{framed}%
\vspace{-1em}%
\begin{verbatim}
$ ./rklt -i inptfile-16bpppb-bigendian-224x512x512.raw \
         -o rkltfile-16bpppb-bigendian-224x512x512.raw \
         -ig 224 512 512 3 0 -og 224 512 512 4 0 \
         -ti side-information.file -D 0 -d 0 \
         -es 0.01 --enableClustering 2 56 1
\end{verbatim}%
\vspace{-1em}%
\end{framed}

\item Perform a lossy KLT and create a JPEG2000-compatible bitstream:
\begin{framed}%
\vspace{-1em}%
\begin{verbatim}
$ ./rklt -i infile.raw -ig $Z $Y $X 3 0 --lossy \
         --dumpEquivalentMatrix KLT_matrix -d 0

$ ./matrix_range_increase.sh "KLT_matrix.klt" $Z 16 \
  | tr '\n' ',' | sed 's/,$/\n/' > "new_range.txt"

$ echo -n "Mvector_coeffs:I2=" > "KLT_Matrix_KDU_param.txt"
$ hexdump -v -e '"%f,"' "$TEMPFOLDER/KLT_matrices.means" \
  | sed 's/,$//' >> "KLT_Matrix_KDU_param.txt"
$ echo -n " Mmatrix_coeffs:I3=" >> "KLT_Matrix_KDU_param.txt"
$ hexdump -v -e '"%f,"' "$TEMPFOLDER/KLT_matrices.klt" \
  | sed 's/,$//' >> "KLT_Matrix_KDU_param.txt"

$ kdu_compress_fixed -quiet Clayers=1 Cycc=no -no_weights \
           -i infile.raw\*$Z@$((X*Y*2)) -o "output.jpx" \
           Creversible=no Mcomponents=$Z Msigned=yes \
           Mprecision=16 Mvector_size:I2=$Z \
           Mmatrix_size:I3=$((Z*Z)) -s "KLT_Matrix_KDU_param.txt" \
           Mstage_inputs:I1=\{0,$((Z-1))\} \
           Mstage_outputs:I1=\{0,$((Z-1))\} \
           Mstage_collections:I1=\{$Z,$Z\} \
           Mstage_xforms:I1=\{MATRIX,3,2,0,0\} Mnum_stages=1 Mstages=1 \
           Sdims=\{$Y,$X\} Sprecision=$(cat "new_range.txt") Ssigned=yes \
           Qstep=0.0000001 -rate $BITRATE
\end{verbatim}%
\vspace{-1em}%
\end{framed}
where \texttt{./matrix\_range\_increase.sh} is:
\begin{framed}%
\vspace{-1em}%
\begin{verbatim}
#!/bin/bash
# Calculates the range increase produced by the KLT
cat "$1" | transpose.pl | hexdump -ve '1/8 "%f\n"' \ 
 | awk 'BEGIN {a=0; b=0; c='"$2"'; d='"$3"'};
    {if ($1>0) a+=$1; else a-=$1; b++;
     if(b == c) {b=0; print int(log(a)/log(2)-2000)+2000+d; a = 0;} };' \
 | sed 's/-inf/1/'
\end{verbatim}%
\vspace{-1em}%
\end{framed}

\textbf{Note:} the read buffer of Kakadu for option files must be enlarged to allow for such a long command line.

\end{itemize}
 %--enableClustering 2 $CSIZE $CMODE $SUBSAMPLING || exit 1;
\section{Notes}

If you need further assistance, you might want to contact us directly.

\bibliographystyle{IEEEtran}
\bibliography{IEEEabrv,biblio}

\end{document}
